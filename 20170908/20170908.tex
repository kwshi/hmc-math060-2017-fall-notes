\documentclass[11pt]{article}

\usepackage{microtype}
\usepackage{parskip}

\usepackage{amsmath, amssymb, amsthm}
\usepackage{mathtools, thmtools}

\usepackage[T1]{fontenc}
\usepackage[ascii]{inputenc}

\usepackage[margin=60pt]{geometry}

\usepackage{fancyhdr}
\usepackage{lastpage}

\usepackage{framed}

\fancypagestyle{notes}{
    \fancyhf{}
    \rhead{Friday, 8 September 2017 \\}
    \chead{\thepage\ / \pageref{LastPage} \\}
    \lhead{Math 60: Multivariable Calculus \\ Directional Derivatives, Gradients, High-Order Derivatives, Hessian Matrices, and Potentials}
    %\rhead{4 September 2017}
}

\newcommand{\uvec}[1]{\mathbf{\hat{#1}}}
\renewcommand{\vec}[1]{\mathbf{#1}}

\usepackage{tikz}
\usepackage{tikz-3dplot}
\usetikzlibrary{calc, intersections}

\pagestyle{notes}

\tdplotsetmaincoords{60}{120}

\begin{document}

\subsection*{Second-order partial derivatives}

\paragraph{Example}

Take \(f(x, y, z) = x^2 y + y^2 z\).

Then the first-order partial derivatives are
\begin{align*}
    \partial_x f &= 2xy, \\
    \partial_y f &= x^2 + 2yz, \\
    \partial_z f &= y^2.
\end{align*}

We can take derivatives again to get \emph{second-order} (and higher order, and so on) partial derivatives:
\begin{align*}
    \partial_x^2 f = \frac{\partial^2 f}{\partial x^2} &= 2y, \\
    \partial_y^2 f &= 2z, \\
    \partial_z^2 f &= 0.
\end{align*}

\subsection*{Mixed partial derivatives}

If some multivariable function \(f \colon \mathbb R^n \to \mathbb R\) has continuous first- and second-order partial derivatives, then its mixed derivatives are the same (regardless of the order in which the partial derivatives are taken):
\[
    \frac{\partial^2 f}{\partial x_{i_1} \, \partial x_{i_2}} = \frac{\partial^2 f}{\partial x_{i_2} \, \partial x_{i_1}}
\]

Thus the Hessian matrix, or the derivative matrix of the gradient vector field, is diagonally symmetric.

\subsection*{Rules of Derivatives}

The rules of derivatives remain the same as in one-variable calculus. 

\paragraph{Product rule}

For functions \(f \colon \mathbb R^n \to \mathbb R\) and \(g \colon \mathbb R^n \to \mathbb R\), the product \(f(\vec x) g(\vec x)\) has derivative matrix
\[
    Df(\vec x) \, g(\vec x) + g(\vec x) \, Df(\vec x).
\]

\subsection*{Potential Functions}

Consider some ``potential function,'' for example, \(f(x, y, z) = xy \sin z\). It has gradient
\[
    \nabla f = (f_x, f_y, f_z) = (y \sin z, x \sin z, x y \cos z).
\]
The potential function \(f \colon \mathbb R^3 \to \mathbb R\) is a \emph{real-valued} function (or \emph{many-to-one}). The gradient \(\nabla f \colon \mathbb R^3 \to \mathbb R^3\), however, is a vector field (a \emph{many-to-many} function). \emph{Whooo, that is so cool.}

\paragraph{Question}

Given a vector field \(\vec F\) (for example, \(\mathbb R^3 \to \mathbb R^3\)). Can this function \(\vec F\) come from some real-valued function \(f\) by taking the gradient \(\nabla f\)? That is, is there some function \(f \colon \mathbb R^3 \to \mathbb R\) such that \(\nabla f = \vec F\)?

Good stuff. This shows up in physics. 

\paragraph{Example}

Take some vector field \(\vec F(x, y, z) \colon \mathbb R^3 \to \mathbb R^3 = (2x, 2y, 2z)\). We happen to notice that this function can be the gradient of 
\[
    f\colon \mathbb R^3 \to \mathbb R = x^2 + y^2 + z^2 \quad (+C),
\]
since
\[
    \nabla f = (2x, 2y, 2z) = \vec F.
\]
Then we can call \(\vec F\) a \emph{gradient vector field}, and \(f\) is a potential function of the vector field \(\vec F\).

\paragraph{Equipotential surfaces}

An \emph{equipotential surface} of a vector field \(\vec F\) is just a \emph{level surface} of \(f\), the (real-valued) potential function of the vector field \(\vec F\), where \(\nabla f = \vec F\).

\paragraph{\emph{How do we find the potential function of a vector field?}}

Consider again the example \(\vec F = (2x, 2y, 2z)\). Suppose there exists some function \(f \colon \mathbb R^3 \to \mathbb R\), such that
\[
    \nabla f = \vec F.
\]
Then 
\[
    (f_x, f_y, f_z) = (2x, 2y, 2z).
\]
Then
\begin{align*}
    f_x &= 2x, \\
    f_y &= 2y, \\
    f_z &= 2z.
\end{align*}
Then we integrate with respect to \(x\),
\[
    f(x, y, z) = x^2 + h_1(y, z),
\]
where \(h_1\) may be a function of \(y\) and \(z\) since, when taking partial derivatives, other variables are held constant. Likewise,
\begin{align*}
    f(x, y, z) &= y^2 + h_2(x, z), \\
    f(x, y, z) &= z^2 + h_3(x, y).
\end{align*}
Now, doing our best to reconcile the three results, we guess a little, piece things together, and conclude something along the lines of 
\[
    f = x^2 + y^2 + z^2 \quad (+C).
\]

Eventually, we'll do something that might be called \emph{path integrals}, but we shouldn't spoil any secrets right now.

\subsection*{Using the gradient to find the tangent plane}

Consider the surface \(f(\vec x) = f(x, y, z)\). Then consider some point \(\vec x\) on the tangent plane to the level surface of \(f\) at a point \(\vec x_0\). Since the gradient is perpendicular to the tangent plane,
\[
    \vec x - \vec {x_0} \perpto \nabla f(\vec {x_0}),
\]
so that
\[
    \nabla f(\vec {x_0}) \cdot (\vec x - \vec{x_0}) = 0 = f_x(\vec {x_0}) (x-x_0) + f_y(\vec {x_0}) (y-y_0) + f_z(\vec {x_0}) (z-z_0).
\]















\end{document}



