\documentclass{multi}

\course{Math 60: Multivariable Calculus}
\title{Spherical coordinates, line integrals}
\date{Wednesday, 2017 September 2017}

\begin{document}

\section*{Spherical coordinates}

In a spherical coordinate system \((\rho, \theta, \phi)\), the spherical volume element is given by
\[
    \diff V = \rho^2 \sin \phi \, \diff \rho \, \diff \phi \, \diff \theta.
\]

% spherical visualization


\paragraph{Example}

\begin{mdframed}
    Find the mass of a cone with density at position \((x, y, z)\) given by
    \[
        f(x, y, z) = \frac{e^{x^2 +y^2+z^2}}{\sqrt{x^2 + y^2 + z^2}}.
    \]
    % shape of cone: vertex at origin, pointing up, 45deg cross section angle
\end{mdframed}

The mass can be found by integrating the density over the volume:
\[
    m = \iiint_W f(x, y, z) \, \diff V.
\]
We write \(f\) in spherical coordinates:
\[
    f(x, y, z) = \frac{e^{x^2 + y^2 + z^2}}{\sqrt{x^2 + y^2 + z^2}} = \frac{e^{\rho^2}}{\rho}.
\]
Then the volume element is
\[
    \diff V = \rho^2 \sin \phi \, \diff \rho \, \diff \phi \, \diff \theta.
\]
Then integrate:
\begin{align*}
    m &= \int_{\theta=0}^{2\pi} \int_{\phi=0}^{\frac \pi 4} \int_{\rho=0}^{2} 
    \frac{e^{\rho^2}}{\rho} \, \rho^2 \sin \phi \, \diff \rho \, \diff \phi \, \diff \theta \\
    &= \iiint \rho e^{\rho^2} \sin \phi \, \diff \rho \, \diff \phi \, \diff \theta \\
    &= 
\end{align*}


In general, it is a good idea to consider using spherical coordinates rather than cartesian coordinates if the function seems to depend on the spherical radius \(\rho^2 = x^2+y^2+z^2\), or if the shape of the region \(W\) lends itself to spherical coordinates.

Likewise, if the integrand function depends on the plane (``cylindrical'') radius \(r^2 = x^2+y^2\), or if the region has some cylindrical symmetry, it may be wise to consider using cylindrical geometry.

\section*{Line integrals}

There are two kinds of line integrals.
\begin{itemize}
\item We may integrate a \emph{real-valued} function \(f \colon \real^n \to \real\) over a curve. We can interpret this integral as the \emph{total mass} of the curve given the \emph{linear density function} \(f\).

\item We may integrate a \emph{vector-valued} function \(\vec F \colon \real^n \to \real^n\) over a curve. This integral shows up a little in physics, where the integral \(\int \vec F \cdot \diff \vec s\) is called the \emph{work} done by the force \(\vec F\).
\end{itemize}

\paragraph{Example}

\begin{mdframed}
Find the mass of the wire \(C\) which is the section of the parabola \(y = x^2\) from \((1, 1)\) to \((2, 4)\), given the linear mass density of the wire \(\rho (x, y, z) = 2x\).
\end{mdframed}

% diagram of parabola section

The total mass \(m\) is found by the integral of the density over the curve:
\[
    m = \int_C \rho \, \diff s = \int_C 2x \, \diff s,
\]
where \(\diff S\) is the ``differential'' of the arc length, or, intuitively, an infinitesimally small section of the arc length. That is,
\[
    \diff s = |\diff \vec x| = \sqrt{\diff x^2 + \diff y^2} = |\dot {\vec x}(t)| \, \diff t,
\]
for some \emph{parametrization} of the curve \(\vec x(t)\) (so that the velocity is \(\dot{\vec x}(t)\), and the speed \(|\dot{\vec x}(t)|\). Then, to parametrize the curve, let
\begin{align*}
    x &= t, \\
    \implies y &= t^2,
\end{align*}
so that the curve \(C\) from \((1, 1)\) to \((2, 4)\) under this parametrization is defined over \(t \in [1, 2]\),
and the speed is
\[
    |(\dot x(t), \dot y(t))| = |(1, 2t)| = \sqrt{1+4t^2},
\]
and the arc-length-differential is then
\[
    \diff s = |(\dot x(t), \dot y(t))| \, \diff t = \sqrt{1+4t^2} \, \diff t.
\]
Then we can integrate for the mass:
\begin{align*}
    m &= \int_C 2x \, \diff s \\
    &= \int_C 2x \sqrt{1+4t^2} \, \diff t \\
    &= \int_C 2t \sqrt{1+4t^2} \, \diff t \\
    &= \left.\frac 1 4 \, \frac 2 3 \, (1+4t^2)^{\frac 3 2}\right\rvert_C \\
    &= \left. \frac 1 6 \, (1+4t^2)^{\frac 3 2}\right\rvert_1^2 \\
    &= \frac 1 6 \, (1+4t^2)
\end{align*}

% examples:
% F = (x, y) and also F = (3/4, 1/2)
% over a circle section defined by r=(cos t, sin t) for t from 0 to pi

% diagrams

% conceptually work equals zero because F perpendicular to r
% or do it



\end{document}

