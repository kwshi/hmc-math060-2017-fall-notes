\documentclass{multi}

\title{Green's theorem!}
\course{Math 60: Multivariable Calculus}
\date{2017 September 29 (Friday)}

\begin{document}

\section*{Green's Theorem}

For some \emph{bounded} region \(D\), Green's theorem gives a relation between a double integration over the region \(D\) and a line integration over the \emph{boundary} \(\partial D\) of \(D\).

\paragraph{Example}

Consider some circular disk \(D\) of radius \(r\).

There are three ways to parametrize the integral over the circular boundary \(\partial D\) in the \emph{counter-clockwise} direction:
\begin{itemize}
\item \(x = R \cos \theta, y = R \sin \theta\), for \(0 \le \theta < 2 \pi\). This parametrization is the standard conversion to polar coordinates. It works.
\item \(x = R \sin \theta, y = R \cos \theta\), for \(0 \le \theta < 2 \pi\). While this parametrization does cover the circular boundary, the \emph{direction} of integration is reversed; this parametrization corresponds to a \emph{clockwise} integration along the circular boundary. Orientation matters.
\item \(x = R \cos 2\pi t, y = R \sin 2 \pi t\), for \(0 \le t < 1\). This parametrization also works, because it traverses the circle with the right orientation.
\end{itemize}

\paragraph{Statement of Green's theorem}

Consider some function in the real plane \(\vec F \colon \real^2 \to \real^2 = (M(x, y), N(x, y))\). Green's theorem relates the integral over the boundary to the integral inside the region:
\[
    \oint_{\partial D} (M \, \diff x + N \, \diff y) = \iint_{D} \left(\frac{\partial N}{\partial x} - \frac{\partial M}{\partial y}\right) \diff x \, \diff y.
\]
Stated in vector form, where \(\vec F = (M, N)\) and \(\diff \vec r = (x, y)\), Green's theorem can easily be thought of as
\[
    \oint_{\partial D} \vec F \cdot \diff \vec r = \iint_{D} (\nabla \times \vec F) \, \diff x \, \diff y.
\]
where \(\nabla \times \vec F\) is thought of as a \emph{two-dimensional curl} of the vector field (nice, Tim!).


\paragraph{Example}

Consider some function \(\vec F = (-y, x)\). We'll still let \(D\) be the circle with radius \(r\). The boundary integral is then
\[
    \oint_{\partial D} \vec F \cdot \diff \vec r = 
    \oint_{\partial D} (-y, x) \cdot (\diff x, \diff y) =
    \oint_{\partial D} (-y \, \diff x + x \, \diff y).
\]
We'll use the parametrization \(x = R \cos \theta, y = R \sin \theta\). Then the integral becomes
\[
    \oint_{\partial D} (-(R \sin \theta) (-R \cos \theta \, \diff \theta) +
                        -(R \cos \theta) (R \sin \theta \, \diff \theta)),
\]
since \(\diff x = R (-\cos \theta \, \diff \theta) = -R \cos \theta \, \diff \theta\), and \(\diff y = R \sin \theta \, \diff \theta\). Simplify the boundary integral:
\[
    \oint_{\partial D} R^2 (\sin^2 \theta + \cos^2 \theta) \, \diff \theta = \oint_{\partial_D} R^2 \, \diff \theta
    = R^2 (2 \pi) = 2 \pi R^2.
\]
But notice, by not much coincidence, that the result of the integral contains a \(\pi R^2\) factor, which looks very much like the \emph{area} of the region! 

Suppose we took the integral of the curl over the surface:
\[
    \iint_D \left(\frac{\partial}{\partial x} (x) - \frac{\partial}{\partial y} (-y)\right) \diff x \, \diff y =
    \iint_D 2 \, \diff x \, \diff y = 2 \pi R^2,
\]
so that the two integrals match up. Wow, Green's theorem is true! What a surprise. {\tiny If only we did proofs in this class\dots}





\end{document}
